\begin{titlepage}
\begin{center}
    \textsc{\LARGE École Polytechnique Fédérale de~Lausanne}\\[1.5cm] 
    {\huge \bfseries Optical Engineering: Monomode Fiber}\\[0.4cm] 
    \begin{tabular}{|p{5cm}|p{4cm}|}
        \hline
        Group & C-8 \\ \hline
        Students & Loïc \textsc{Amez-Droz} \newline Florian \textsc{Reinhard} \\ \hline
        Date of lecture & 20.03.2015 \\ \hline
        Date of final report return & 27.03.2015 \\ \hline
    \end{tabular}
\end{center}


\begin{abstract}
    In this study, we analyse the transmission properties of monomode fibers.
    First we observe transmission light modes using a laser $\left( \lambda = \SI{633}{\nano\meter} \right)$ and the \emph{P1-SMF28E-FC-2} $\left( \lambda = \SI{1260}{\nano\meter} \mbox{ to } \SI{1625}{\nano\meter} \right)$.
    The different polarizations of a mode are perceptible with a polarizer in different orientations.
    In the second experience we determinate the numerical aperture of the \emph{P1-630A-FC} fiber $\left( \mbox{NA} = 0.11 \right)$.
    Then we measure the transmitted intensity for two numerical apertures.
    The transmission ratio $r = 1.3$ in favor of the lower NA.
    Finally we compare the coupling of three different light sources (halogen, LED, laser) into the \emph{P1-630A-FC} fiber.
\end{abstract}
 
\vfill
\end{titlepage}

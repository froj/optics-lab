\begin{titlepage}
\begin{center}
    \textsc{\LARGE École Polytechnique Fédérale de~Lausanne}\\[1.5cm] 
    {\huge \bfseries Optical Engineering: Detector and Noise}\\[0.4cm] 
    \begin{tabular}{|p{5cm}|p{4cm}|}
        \hline
        Group & C \\ \hline
        Students & Loïc \textsc{Amez-Droz} \newline Florian \textsc{Reinhard} \\ \hline
        Date of lecture & 27.02.2015 \\ \hline
        Date of final report return & 06.03.2015 \\ \hline
    \end{tabular}
\end{center}


\begin{abstract}
First, we interested in the black noise caused by the thermal electron generation and the increasing of the noise by amplification.
We analyze the noise reduction by averaging in the case of low and high gain.
Then we determine how many columns with high gain are needed to limit the noise to the amount observed for a single column with low gain.
Comparing a homogeneous part of a picture taken with low and high gain, we assess the dynamic range of the sensor.
Finally we study \emph{high dynamic range} (HDR) imaging.
It is used to show details on a normally saturated or underexposed zone by assembling the same pictures taken with different exposures.
\end{abstract}
 
\vfill
\end{titlepage}

\begin{titlepage}
\begin{center}
    \textsc{\LARGE École Polytechnique Fédérale de~Lausanne}\\[1.5cm] 
    {\huge \bfseries Optical Engineering: Interferometer}\\[0.4cm] 
    \begin{tabular}{|p{5cm}|p{4cm}|}
        \hline
        Group & C-8 \\ \hline
        Students & Loïc \textsc{Amez-Droz} \newline Florian \textsc{Reinhard} \\ \hline
        Date of lecture & 17.04.2015 \\ \hline
        Date of final report return & 24.04.2015 \\ \hline
    \end{tabular}
\end{center}


\begin{abstract}
    In this work, we analyze the properties of the Michelson interferometer.
    Separating the ray in two and recombining it we obtain an interference figure which yields information about the position of two virtual sources.
    When they are both on the same optical axis, there is \emph{zero optical paht difference}.
    The contrast between low and high at the ZOPD point is $C = 0.39$.
    Then we determinate the laser source spectral width observing the beat phenomenon in the contrast measure ($\Delta \lambda = \SI{0.22}{\nano\meter}$).
    Finally we determinate the incline angle corresponding to \SI{180}{\degree} phase shifting performing a pressure on the optical support ($\SI{0.06e-3}{\degree}$).
\end{abstract}
 
\vfill
\end{titlepage}

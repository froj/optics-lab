\begin{titlepage}
\begin{center}
    \textsc{\LARGE École Polytechnique Fédérale de~Lausanne}\\[1.5cm] 
    {\huge \bfseries Optical Engineering: Multimode Fibre}\\[0.4cm] 
    \begin{tabular}{|p{5cm}|p{4cm}|}
        \hline
        Group & C-8 \\ \hline
        Students & Loïc \textsc{Amez-Droz} \newline Florian \textsc{Reinhard} \\ \hline
        Date of lecture & 13.03.2015 \\ \hline
        Date of final report return & 20.03.2015 \\ \hline
    \end{tabular}
\end{center}


\begin{abstract}
First we determine the numerical aperture of the fiber using a collimated laser searching the angle of acceptance (NA = 0.5).
In the second experience we calculate the NA measuring at different position the diameter of the point on the sensor (NA = 0.36).
Then we compare the transmission properties of the fiber with different sources of light (halogen, LED, laser).
Finally we measure the injection efficiency of a LED source with a large and a small numerical aperture ($r = 6.244$ and  $r = 2.516$ without dark noise correction).
\end{abstract}
 
\vfill
\end{titlepage}

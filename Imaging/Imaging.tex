% Déclaration du type de document (report, book, paper, etc...)
\documentclass[a4paper, 12pt]{paper} 
 
% Package pour avoir Latex en français
\usepackage[utf8]{inputenc}
\usepackage[T1]{fontenc}
 
% Quelques packages utiles
\usepackage{listings} % Pour afficher des listings de programmes
\usepackage{graphicx} % Pour afficher des figures
\usepackage{amsthm}   % Pour créer des théorèmes et des définitions
\usepackage{amsmath}
\usepackage{microtype} % Optical margins FTW
\usepackage{url}
\usepackage{booktabs} % Allows the use of \toprule, \midrule and \bottomrule in tables for horizontal lines
\usepackage[per-mode=symbol]{siunitx}
\usepackage{floatrow}
\usepackage{caption}
\usepackage{subcaption}
\usepackage{fullpage}
\usepackage{lipsum}



\author{Loïc Amez-Droz \and Florian Reinhard}
\title{Imaging}

% Début du document
\begin{document}
\begin{titlepage}
\begin{center}
    \textsc{\LARGE École Polytechnique Fédérale de~Lausanne}\\[1.5cm] 
    {\huge \bfseries Optical Engineering: Monomode Fiber}\\[0.4cm] 
    \begin{tabular}{|p{5cm}|p{4cm}|}
        \hline
        Group & C-8 \\ \hline
        Students & Loïc \textsc{Amez-Droz} \newline Florian \textsc{Reinhard} \\ \hline
        Date of lecture & 20.03.2015 \\ \hline
        Date of final report return & 27.03.2015 \\ \hline
    \end{tabular}
\end{center}


\begin{abstract}
    In this study, we analyse the transmission properties of monomode fibers.
    First we observe transmission light modes using a laser $\left( \lambda = \SI{633}{\nano\meter} \right)$ and the \emph{P1-SMF28E-FC-2} $\left( \lambda = \SI{1260}{\nano\meter} \mbox{ to } \SI{1625}{\nano\meter} \right)$.
    The different polarizations of a mode are perceptible with a polarizer in different orientations.
    In the second experience we determinate the numerical aperture of the \emph{P1-630A-FC} fiber $\left( \mbox{NA} = 0.11 \right)$.
    Then we measure the transmitted intensity for two numerical apertures.
    The transmission ratio $r = 1.3$ in favor of the lower NA.
    Finally we compare the coupling of three different light sources (halogen, LED, laser) into the \emph{P1-630A-FC} fiber.
\end{abstract}
 
\vfill
\end{titlepage}

\section{Procedures and results}
\subsection{Saturation and intensity adjustment of the camera}
\label{sec:saturation}

The camera can auto-adjust the exposure and gain.
The sensor saturates in this mode at the windows and at the reflections on the curtains~(figure~\ref{fig:auto}).
Reducing manually the exposure decreases the amount of saturation and the colors become perceptible.
The blue sky is visible on the picture without saturations~(figure~\ref{fig:no_sat}).
We noticed also that details under the desks are lost in the darkness.
We have to do some compromises between saturations and obscurity.
The intensity is studied on a horizontal line of pixels.
The saturation means that he value of the intensity exceeds the sensor's limit of 255.

\begin{figure}[h]
    \centering
    \begin{subfigure}[p]{0.40\textwidth}
        \includegraphics[width=\textwidth]{img/auto}
        \caption{Automatic exposure and gain.}
\label{fig:auto}
    \end{subfigure}
    \begin{subfigure}[p]{0.40\textwidth}
        \includegraphics[width=\textwidth]{img/auto_int}
        \caption{Intensity plot at $y=300$.}
    \end{subfigure}
    \begin{subfigure}[p]{0.40\textwidth}
        \includegraphics[width=\textwidth]{img/little_sat}
        \caption{Only a couple of pixels are saturated.}
    \end{subfigure}
    \begin{subfigure}[p]{0.40\textwidth}
        \includegraphics[width=\textwidth]{img/little_int}
        \caption{Intensity plot at $y=300$.}
    \end{subfigure}
    \begin{subfigure}[p]{0.40\textwidth}
        \includegraphics[width=\textwidth]{img/no_sat}
        \caption{No saturation at all.}
\label{fig:no_sat}
    \end{subfigure}
    \begin{subfigure}[p]{0.40\textwidth}
        \includegraphics[width=\textwidth]{img/no_sat_int}
        \caption{Intensity plot at $y=300$.}
    \end{subfigure}
    \caption{Comparison of different exposure times.}
\label{fig:saturation}
\end{figure}


\subsection{Procedure to measure the focal length}

First, we measure the magnification at two different focus settings.
For that, we take pictures of a ruler, which gives us the \emph{object size} $s_O$~(figure~\ref{fig:focus_length}).
The \emph{image size} $s_I$ is constant and given by size of the camera sensor.
We then use equation~\ref{equ:magnification} to calculate the magnification.

\begin{figure}[H]
    \centering
    \begin{subfigure}[b]{0.45\textwidth}
        \includegraphics[width=\textwidth]{img/focale1.jpg}
        \caption{Object size of \SI{96}{\milli\meter}}
    \end{subfigure}
    \begin{subfigure}[b]{0.45\textwidth}
        \includegraphics[width=\textwidth]{img/focale2.jpg}
        \caption{Object size of \SI{35.5}{\milli\meter}}
    \end{subfigure}
    \caption{Focusing on a ruler to know the object size.}
\label{fig:focus_length}
\end{figure}

\begin{equation}
    m = \frac{s_I}{s_O}
    \label{equ:magnification}
\end{equation}

We consider the measurement error of the image size $\Delta s_I$ negligible, and calculate the error of the magnification using equation~\ref{equ:magnification_err}.

\begin{equation}
    \Delta m = \frac{s_I}{s_O^2} \Delta s_O
    \label{equ:magnification_err}
\end{equation}

\begin{table}[H]
    \centering
    \begin{tabular}{c S[table-format=1.1] S[table-format=1.3] S[table-format=1.4] S[table-format=1.2] c S[table-format=1.2] S[table-format=1.2]}
        \toprule
        {No.} & {Object size $s_O$} & {Image size $s_I$} & {$m$} & {$\Delta s_O$} & {$\Delta s_I$} & {$\Delta m$} & {$\frac{\Delta m}{m}$} \\
        \midrule
        A & \SI{96.0}{\milli\meter} & \SI{4.535}{\milli\meter} & \num{0.0472} & \SI{0.25}{\milli\meter} & $-$ & \num{1.23e-4} & 0.25\% \\
        B & \SI{35.5}{\milli\meter} & \SI{4.535}{\milli\meter} & \num{0.128} & \SI{0.25}{\milli\meter} & $-$ & \num{9.00e-4} & 0.70\% \\
        \bottomrule
    \end{tabular}
    \caption{Magnifications for two different foci and their errors.}
\label{tab:magnification}
\end{table}


The focal length is given by equation~\ref{equ:focal_length} with $ d_{IB} - d_{IA} = 0.333 mm $ being the distance we moved the objective (corresponding to one full turn).

\begin{equation}
    f = \frac{d_{IB} - d_{IA}}{m_B - m_A} = \SI{4.1}{\milli\meter}
    \label{equ:focal_length}
\end{equation}

Considering we screwed the objective a full turn with a precision of \SI{10}{\degree} then $\Delta \left(d_{IB} - d_{IA}\right) = \SI{9.25}{\micro\meter}$.

\begin{equation}
    \Delta f = \frac{1}{m_B - m_A} \Delta \left(d_{IB} - d_{IA}\right)
    + \frac{d_{IB} - d_{IA}}{{\left(m_B - m_A\right)}^2}
        \left(\Delta m_A + \Delta m_B\right) = \SI{0.167}{\milli\meter}
    \label{equ:focal_length_err}
\end{equation}

\begin{equation}
    \frac{\Delta f}{f} = \SI{4.0}{\percent}
    \label{equ:focal_length_err_percent}
\end{equation}

\subsection{Measurement of the field of view}
We measured the object size $s_O = \SI{54.5}{\milli\meter}$ and the distance to the objective $d_O = \SI{46.0}{\milli\meter}$ with a ruler.
Simple trigonometry yields equation~\ref{equ:view_angle}.

\begin{equation}
    \alpha = 2 \arctan{\frac{s_O}{2 d_O}} = \SI{61.3}{\degree}
    \label{equ:view_angle}
\end{equation}

We consider $\Delta s_O = \SI{0.25}{\milli\meter}$ and $\Delta d_O = \SI{0.5}{\milli\meter}$ and calculate the error with equation~\ref{equ:view_angle_err}.

\begin{equation}
    \Delta \alpha = \frac{4 d_O}{s_O^2 + 4 d_O^2} \Delta s_O
    + \frac{4 s_O}{s_O^2 + 4 d_O^2} \Delta d_O = \SI{0.777}{\degree}
    \label{equ:view_angle_err}
\end{equation}

\begin{equation}
    \frac{\Delta \alpha}{\alpha} = \SI{1.27}{\percent}
    \label{equ:view_angle_err_percent}
\end{equation}

\subsection{Measurement of the F\# number}
Using a laser beam collimated by a focal lens, we determine the F\# number by measuring the diameter of the laser spot on the pictures we took with the camera.
The diameter's variation with a change of focus is used to calculate the angle of acceptance of the system.
This angle allows us to get the \emph{numerical aperture} (NA) and the F\# number.
The diameter of the spot is measured by counting pixels.
We know that the image covers 1600 pixels and is \SI{4.536}{\milli\meter} wide.
With this method we can limit the error of the spot diameter to 3 pixels (or \SI{9}{\micro\meter}) and the error of the relative focus distance to \SI{10}{\degree}, which corresponds to \SI{9.25}{\micro\meter}.

\begin{figure}[H]
    \centering
    \begin{subfigure}[b]{0.45\textwidth}
        \includegraphics[width=\textwidth]{img/focus_cropped}
        \caption{Spot of the focused laser beam.}
    \end{subfigure}
    \begin{subfigure}[b]{0.45\textwidth}
        \includegraphics[width=\textwidth]{img/050_cropped}
        \caption{Spot after defocusing by half a turn.}
    \end{subfigure}
    \caption{Measuring the spot size to calculate the acceptance angle.}
\label{fig:spot}
\end{figure}


\begin{table}[h]
    \centering
    \begin{tabular}{S[table-format=1.2] S[table-format=1.3] S[table-format=2] S[table-format=1.3]}
        \toprule
        {Turn} & {Relative focus distance $z_i$ (\SI{}{\milli\meter})} & {Spot diameter $d_i$ (px)} & {(\SI{}{\milli\meter})} \\
        \midrule
        0    & 0     &  5 & 0.014 \\
        0.25 & 0.083 & 16 & 0.045 \\
        0.5  & 0.167 & 30 & 0.085 \\
        0.75 & 0.250 & 43 & 0.122 \\
        1    & 0.333 & 57 & 0.162 \\
        1.25 & 0.416 & 73 & 0.207 \\

        \bottomrule
    \end{tabular}
    \caption{Measurements taken to determine the F\# number.}
\label{tab:f_number}
\end{table}

A linear regression of the values in table~\ref{tab:f_number} yields $ \frac{d-d_0}{z} \approx 0.463 $~(figure~\ref{fig:f_number}).

\begin{figure}[H]
    \centering
    \includegraphics[width=0.8\textwidth]{img/f_number.png}
    \caption{Diameters of the spots measured for different distances to the focal point.}
\label{fig:f_number}
\end{figure}

\begin{equation}
    u = \arctan{\frac{d - d_0}{2 z}} = \SI{13}{\degree}
    \label{equ:acceptance_angle}
\end{equation}

We use the measurement with the furthest relative focus distance and the greatest spot diameter to calculate the maximal error.

\begin{equation}
    \Delta u = \frac{z_5}{{\left(d_5-d_0\right)}^2 + z_5^2} 2 \Delta \left(d_5-d_0\right)
    + \frac{d_5-d_0}{{\left(d_5-d_0\right)}^2 + z_5^2} \Delta z_5 = \SI{2.41}{\degree}
    \label{equ:acceptance_angle_err}
\end{equation}

\begin{equation}
    \frac{\Delta u}{u} = \SI{18.6}{\percent}
    \label{equ:acceptance_angle_err_percent}
\end{equation}

\begin{equation}
    \mbox{NA} = n \sin{u} = 0.23
    \label{equ:numerical_aperture}
\end{equation}

\begin{equation}
    \Delta \mbox{NA} = \cos{u} \Delta u = 0.041
    \label{equ:numerical_aperture_err}
\end{equation}

\begin{equation}
    \mbox{F\#} = \frac{1}{2 \mbox{NA}} = 2.2
    \label{equ:f_number}
\end{equation}

\begin{equation}
    \frac{\Delta \mbox{F\#}}{\mbox{F\#}} = \frac{\Delta \mbox{NA}}{\mbox{NA}}
        = \SI{18.1}{\percent}
    \label{equ:f_number_err_precent}
\end{equation}

\subsection{Example from real world}

Large aperture objectives can take pictures in relatively dark conditions without flash. 
The \emph{Canon EF 50mm f/1.2L USM} is an example of it. 
His large aperture allows the selection of a precise depth of field while blurring the rest~(figure~\ref{fig:real_world}).

\begin{figure}[H]
    \centering
    \includegraphics[width=0.7\textwidth]{img/real_world.jpg}
    \caption{A picture taken with the \emph{Canon EF 50mm f/1.2L USM} objective showing the very distinct depth of field.}
\label{fig:real_world}
\end{figure}

\section{Discussion and conclusions}

In Section~\ref{sec:saturation} we saw that maladjusted exposure time and gain can lead to information loss.
Further, we could measure optical properties of a real life optical system and experience the difficulties thereof.
The estimation of the error revealed that it is quite easy to determine a precise value for the \emph{angle of view}, but it takes some more effort for the \emph{focal length}.
The general formula for linear error propagation didn't yield realistic results for the angle of acceptance, as it was determined by a linear regression of a set of values.

\end{document}

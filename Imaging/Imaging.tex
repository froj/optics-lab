% Déclaration du type de document (report, book, paper, etc...)
\documentclass[a4paper, 12pt]{paper} 
 
% Package pour avoir Latex en français
\usepackage[utf8]{inputenc}
\usepackage[T1]{fontenc}
 
% Quelques packages utiles
\usepackage{listings} % Pour afficher des listings de programmes
\usepackage{graphicx} % Pour afficher des figures
\usepackage{amsthm}   % Pour créer des théorèmes et des définitions
\usepackage{amsmath}
\usepackage{microtype} % Optical margins FTW
\usepackage{url}
\usepackage{booktabs} % Allows the use of \toprule, \midrule and \bottomrule in tables for horizontal lines
\usepackage[per-mode=symbol]{siunitx}
\usepackage{floatrow}
\usepackage{caption}
\usepackage{subcaption}
\usepackage{fullpage}
\usepackage{lipsum}



\author{Loïc Amez-Droz \and Florian Reinhard}
\title{Imaging}

% Début du document
\begin{document}
\begin{titlepage}
\begin{center}
    \textsc{\LARGE École Polytechnique Fédérale de~Lausanne}\\[1.5cm] 
    {\huge \bfseries Optical Engineering: Monomode Fiber}\\[0.4cm] 
    \begin{tabular}{|p{5cm}|p{4cm}|}
        \hline
        Group & C-8 \\ \hline
        Students & Loïc \textsc{Amez-Droz} \newline Florian \textsc{Reinhard} \\ \hline
        Date of lecture & 20.03.2015 \\ \hline
        Date of final report return & 27.03.2015 \\ \hline
    \end{tabular}
\end{center}


\begin{abstract}
    In this study, we analyse the transmission properties of monomode fibers.
    First we observe transmission light modes using a laser $\left( \lambda = \SI{633}{\nano\meter} \right)$ and the \emph{P1-SMF28E-FC-2} $\left( \lambda = \SI{1260}{\nano\meter} \mbox{ to } \SI{1625}{\nano\meter} \right)$.
    The different polarizations of a mode are perceptible with a polarizer in different orientations.
    In the second experience we determinate the numerical aperture of the \emph{P1-630A-FC} fiber $\left( \mbox{NA} = 0.11 \right)$.
    Then we measure the transmitted intensity for two numerical apertures.
    The transmission ratio $r = 1.3$ in favor of the lower NA.
    Finally we compare the coupling of three different light sources (halogen, LED, laser) into the \emph{P1-630A-FC} fiber.
\end{abstract}
 
\vfill
\end{titlepage}

\section{Procedures and results}
\subsection{Saturation and intensity adjustment of the camera}
\subsection{Procedure to measure the focal length}

First, we measure the magnification at two different focus settings.
For that, we take pictures of a ruler, which gives us the \emph{object size} $s_O$~(figure~\ref{fig:focus_length}).
The \emph{image size} $s_I$ is constant and given by size of the camera sensor.
We then use equation~\ref{equ:magnification} to calculate the magnification.

\begin{figure}[h]
    \centering
    \begin{subfigure}[b]{0.45\textwidth}
        \includegraphics[width=\textwidth]{img/focale1.jpg}
        \caption{Object size of \SI{96}{\milli\meter}}
    \end{subfigure}
    \begin{subfigure}[b]{0.45\textwidth}
        \includegraphics[width=\textwidth]{img/focale2.jpg}
        \caption{Object size of \SI{35.5}{\milli\meter}}
    \end{subfigure}
    \caption{Focusing on a ruler to know the object size.}
\label{fig:focus_length}
\end{figure}

\begin{equation}
    m = \frac{s_I}{s_O}
    \label{equ:magnification}
\end{equation}

We consider the measurement error of the image size $\Delta s_I$ negligible, and calculate the error of the magnification using equation~\ref{equ:magnification_err}.

\begin{equation}
    \Delta m = \frac{s_I}{s_O^2} \Delta s_O
    \label{equ:magnification_err}
\end{equation}

\begin{table}[h]
    \centering
    \begin{tabular}{c r r r r c r r}
        \toprule
        No. & Object size $s_O$ & Image size $s_I$ & $m$ & $\Delta s_O$ & $\Delta s_I$ & $\Delta m$ & $\frac{\Delta m}{m}$ \\
        \midrule
        A & \SI{96}{\milli\meter} & \SI{4.535}{\milli\meter} & \num{0.0472} & \SI{0.25}{\milli\meter} & - & \num{1.23e-4} & 0.25\% \\
        B & \SI{35.5}{\milli\meter} & \SI{4.535}{\milli\meter} & \num{0.128} & \SI{0.25}{\milli\meter} & - & \num{9.00e-4} & 0.70\% \\
        \bottomrule
    \end{tabular}
    \caption{Magnifications for two different foci and their errors.}
\label{tab:magnification}
\end{table}


The focal length is given by equation~\ref{equ:focal_length} with $ d_{IB} - d_{IA} = 0.333 mm $ being the distance we moved the objective (corresponding to one full turn).

\begin{equation}
    f = \frac{d_{IB} - d_{IA}}{m_B - m_A} = \SI{4.1}{\milli\meter}
    \label{equ:focal_length}
\end{equation}

Considering we screwed the objective a full turn with a precision of \SI{10}{\degree} then $\Delta \left(d_{IB} - d_{IA}\right) = \SI{9.25e-3}{\milli\meter}$.

\begin{equation}
    \Delta f = \frac{1}{m_B - m_A} \Delta \left(d_{IB} - d_{IA}\right)
        + \frac{d_{IB} - d_{IA}}{\left(m_B - m_A\right)^2}
        \left(\Delta m_A + \Delta m_B\right) = \SI{0.167}{\milli\meter}
    \label{equ:focal_length_err}
\end{equation}

\begin{equation}
    \frac{\Delta f}{f} = \SI{4.0}{\percent}
    \label{equ:focal_length_err_percent}
\end{equation}

\subsection{Measurement of the field of view}
We measured the object size $s_O = \SI{54.5}{\milli\meter}$ and the distance to the objective $d_O = \SI{46}{\milli\meter}$ with a ruler.
Simple trigonometry yeilds equation~\ref{equ:view_angle}.

\begin{equation}
    \alpha = 2 \arctan{\frac{s_O}{2 d_O}} = \SI{61.3}{\degree}
    \label{equ:view_angle}
\end{equation}

We consider $\Delta s_O = \SI{0.25}{\milli\meter}$ and $\Delta d_O = \SI{0.5}{\milli\meter}$ and calculate the error with equation~\ref{equ:view_angle_err}.

\begin{equation}
    \Delta \alpha = \frac{4 d_O}{s_O^2 + 4 d_O^2} \Delta s_O
    + \frac{4 s_O}{s_O^2 + 4 d_O^2} \Delta d_O = \SI{0.777}{\degree}
    \label{equ:view_angle_err}
\end{equation}

\begin{equation}
    \frac{\Delta \alpha}{\alpha} = \SI{1.27}{\percent}
    \label{equ:view_angle_err_percent}
\end{equation}

\subsection{Measurement of the F\# number}
\subsection{Example from real world}


\section{Discussion and conclusions}
\lipsum[6]
\end{document}
